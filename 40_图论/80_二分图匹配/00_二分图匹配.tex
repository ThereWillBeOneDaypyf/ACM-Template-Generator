1)一个二分图中的最大匹配数等于这个图中的最小点覆盖数

König 定理是一个二分图中很重要的定理,它的意思是,一个二分图中的最大匹配数等于这个图中的最小点覆盖数。如果你还不知道什么是最小点覆盖,我也在这里说一下:假如选了一个点就相当于覆盖了以它为端点的所有边,你需要选择最少的点来覆盖所有的边。

2)最小路径覆盖=|G|-最大匹配数

在一个 N*N 的有向图中,路径覆盖就是在图中找一些路经,使之覆盖了图中的所有顶点,且任何一个顶点有且只有一条路径与之关联;

(如果把这些路径中的每条路径从它的起始点走到它的终点,那么恰好可以经过图中的每个顶点一次且仅一次);如果不考虑图中存在回路,那么每每条路径就是一个弱连通子集.

由上面可以得出:

1.一个单独的顶点是一条路径;

2.如果存在一路径 $p_1,p_2,......p_k$,其中 $p_1$ 为起点,$p_k$ 为终点,那么在覆盖图中,顶点 $p_1,p2,......p_k$ 不再与其它的顶点之间存在有向边.

最小路径覆盖就是找出最小的路径条数,使之成为 G 的一个路径覆盖.

路径覆盖与二分图匹配的关系:最小路径覆盖=|G|-最大匹配数;

3)二分图最大独立集=顶点数-二分图最大匹配

独立集:图中任意两个顶点都不相连的顶点集合。
